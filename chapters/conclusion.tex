\onehalfspacing
Stress inversion from geological or geophysical observations is a nonlinear problem that can be best solved by the methods that are independent of any linearizing assumption. The GA is a nonlinear approach that searches the global optimum from a population of points rather than starting from a single point. It uses stochastic sampling and selection rules, which enable the algorithm to work in variety of environments, including noisy ones. The algorithm always converges and the solutions stabilize with increasing iterations. Finally, it is a flexible algorithm that offers user defined choices for selection, crossover, iterations, and the probability of each event. Tests on synthetic, noisy and natural fault-slip observations validate the application of the GA. Its scope can be easily extended to other stress indicators requiring inversion of observations for determination of stress states in the Earth`s crust.

Estimating the number of phases in a polyphase fault-slip data has always been a significant problem, and till date, most scientists use cross cutting relationships in the field to determine the polyphase faulting events. With the advancement in numerical techniques, a number of techniques have been developed for the separation of fault-slip data. We have devised a Genetic Algorithm based Polyphase Separation technique to separate and invert the fault-slip data. This algorithm is able to determine the number of phases of faulting, which has been a major issue in most of the existing methods. Another important advantage of the GAPS is its ability to reach the globally optimum solution, which is a major limitation of most iterative methods. Testing the GAPS on data sets with similar orientations, and on data set with uneven number of subsets, shows the robust nature of the algorithm.