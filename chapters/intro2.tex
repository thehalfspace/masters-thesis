\onehalfspacing
Typical inversion of fault-slip observations involves minimizing the sum of angular misfits between the observed and the theoretical slip direction to yield a best fit stress tensor. A number of methods have been proposed for this task (Carey and Brunier, 1974; Angelier, 1979; Angelier et al., 1982; Michael, 1984; Xu, 2004). These methods work well when it is known that all the observations belong to a single stress regime. Quite often, we have multiple phases of faulting occurring in nature, caused by successive events, where each phase is characterized by its own stress tensor. In the region of polyphase faulting, where more than one stress tensor is at play, these inversion techniques will yield meaningless results. It is one of the most challenging tasks for a geologist to separate the different phases of faulting before inverting the observations to find the stress tensor. 

In order to separate the polyphase fault-slip observations into single phase subsets, a variety of methods have been proposed. These methods can be broadly grouped into search methods (Hardcastle and Hills, 1991; Shan et al., 2007) and clustering methods (Nemcok and Lisle, 1995; Yamaji, 2000; Shan et al., 2003). Search methods use exhaustive grid search to group the observations into subsets based on a deviation threshold (Hardcastle and Hills, 1991). This threshold is defined by the user, hence the division of the observations is subject to bias. Furthermore, these methods have difficulty in separating stress tensors with similar orientations. Clustering methods try to group similar faults-slip observations together based on similarity of fault-slip data and/or stress vectors. The Multiple Inverse Method (Yamaji, 2000) inverts every possible quadruplet in the data set, and then groups the data with similar tensors. The Cluster Analysis method (Nemcok and Lisle, 1995) follows a hierarchical clustering approach which uses a similarity coefficient based on a range of trial stress tensors to produce a dendogram. The major limitation with existing clustering methods is that they require visual inference from the user in order to decipher the phases in an unknown fault-slip data set.

Fry (1999) showed that stresses are linearly separable in a six dimensional hyperplane. A number of methods have built upon this formulation of the stress inversion problem to separate polyphase fault-slip data sets (Shan et al., 2003; Shan and Fry, 2005; Hansen et al., 2015). Shan and Fry (2006) pointed out that these methods may only give correct results if the stress associated with the smallest eigenvector is unique. Furthermore, methods using iterative procedures (Shan et al. 2003) have difficulty in overcoming locally optimized division. Shan et al. (2007) have given a method to overcome this limitation, using a grid search approach. This approach is very time consuming and the approach taken for obtaining the optimum number of subsets is an empirical approach and lacks a theoretical basis. For a detailed review of these existing methods and their shortcomings, the reader is directed to (Nemcok et al., 1999; Shan et al., 2003; Leisa and Lisle 2004).

To sum up, two important problems in the existing methods are: (i) finding out the number of phases in the given data, and (ii) difficulty in reaching the globally optimized solution. Some methods circumvent the problem of optimal division of subsets using fuzzy clustering (e.g. Shan et al., 2004), while other methods qualitatively estimate the best division (e.g. Nemcok and Lisle, 1995; Yamaji, 2000).
We design a Genetic Algorithm based Polyphase Separation (GAPS) technique that separates polyphase fault-slip observations from a set of stress tensors, based on the misfit criteria. The GAPS is a robust heuristic search technique that separates the fault-slip data without any assumption about the number of stress states. It reaches the globally optimum solution by using stochastic search and selection rules.
