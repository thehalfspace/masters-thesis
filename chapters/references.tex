%\begin{enumerate}
\footnotesize
{
\setlength{\parindent}{0cm}
\begin{hangparas}{.25in}{1}
    
    Anderson, E.M., 1951, The Dynamics of Faulting: Oliver \& Boyd, Edinburgh, p. 231–248.
    
    Angelier, J., 1975, Sur l’analyse de measures recueillies dans des sites failles: l’utilite d’une confrontation entre les methods dynamiques et cinematiques: Comptes Rendus de l’Academie des Sciences, Paris, v. D281, p. 1805–1808.
    
    Angelier, J., 1979, Determination of the mean principal directions of stresses for a given fault population: Tectonophysics, v. 56, p. T17–T26.
    
    Angelier, J., 1984, Tectonic analysis of fault slip data sets: Journal of Geophysical Research, v. 89, p. 5835–5848.
    
    Angelier, J., 1990, Inversion of field data in fault tectonics to obtain the regional stress—III. A new rapid direct inversion method by analytical means: Geophysical Journal International, v. 103, p. 363–376.
    Angelier, J., 1994. Fault slip analysis and palaeostress reconstruction, in Hancock, P.L., ed., Continental Deformation: Pergamon Press, Oxford, p. 53–100.

    Angelier, J., Tarantola, A., Valette, B., and Manoussis, S. 1982, Inversion of field data in fault tectonics to obtain the regional stress – I. Single phase fault populations: a new method of computing the stress tensor: Geophysical Journal of the Royal Astronomical Society, v. 69, p. 607–621.

    Barker, P., 1999, Genetic Algorithms and their use in Geophysical Problems, Ernest Orlando Lawrence Berkeley National Laboratory [PhD Thesis]: Berkeley, University of California, p. 61–66.

    Bishop, A., W., 1966, The strength of soils as engineering materials – 6th Rankine lectures: Geotechnique, v. 16, p. 91–130.

    Bott, M.H.P., 1959, The mechanics of oblique slip faulting: Geological Magazine, v. 96, p. 109–117.

    Byerlee, J.D., 1978, Friction of rocks: Pure and Applied Geophysics, v. 116, p. 615–626.

    Carey, E., and Brunier, B., 1974, Analyse théorique et numérique d'un modèle mécanique élémentaire appliqué à l'étude d'une population de failles: C. R. Academy of Science, Paris, D, v. 279, p. 891–894.

    Célérier, B., 1988, How much does slip on reactivated fault plane constrain the stress tensor?: Tectonics, v. 7, p.1257–1278.
    
    Celerier, B., Etchecopar, A., Bergerat, F., Vergely, P., Arthaud, F., and Laurent, P., 2012, Inferring stress from faulting: from early concepts to inverse methods: Tectonophysics, 581, 206-219.

    Chop, P., Y., Angelier, J., and Souffache, B., 1996, Distribution of angular misfits in fault-slip data: Journal of Structural Geology, v. 18, p. 1353–1367.

    Delvaux, D., and Sperner, B. 2003, Stress tensor inversion from fault kinematic indicators and focal mechanism data: the TENSOR program. In: new insights into Structural Interpretation and Modelling (D. Nieuwland Ed.): Geological Society, London, Special Publications, 212:75–100.

    Engelder, T., 1993, Stress regimes in the lithosphere: Princeton, NJ, Princeton University Press, 457 p.

    Etchecopar, A., Vasseur, G., and Daignieres, M., 1981, An inverse problem in microtectonics for the determination of stress tensors from fault striation analysis: Journal of Structural Geology v. 14, p. 1121–1131.
    
    Fry, N., 1999, Striated faults: visual appreciation of their constraint on possible paleostress tensors: Journal of Structural Geology, v. 21, p. 7-21. 

    Gephart, J.W., and Forsyth, D.W., 1984, An improved method for determining the regional stress tensor using earthquake focal mechanism data: application to the San Fernando earthquake sequence: Journal of Geophysical Research, v. 89, p. 9305–9320.

    Gerstoft, P., 1994, Inversion of seismoacoustic data using genetic algorithms and aposteriori probability distributions: The Journal of the Acoustical Society of America, v. 95, p. 770–782.

    Goldberg, D.E., 1989, GAs in Search, Optimization and Machine Learning: Kluwer Academic Publishers, Boston, MA, p. 1–40.
    
    Hansen, J. A., Bergh, S. G., Osmundsen, P. T., and Redfield, T. F., 2015. Stress inversion of heterogeneous fault-slip data with unknown slip sense: An objective function algorithm contouring method: Journal of Structural Geology, v. 70, p. 119-140.

    Hardebeck, J.L., and Hauksson, E., 2001, Stress orientations obtained from earthquake focal mechanisms: what are appropriate uncertainty estimates?: Bulletin of the Seismological Society of America, v. 91, p. 250–262.

    Hardcastle, K.C., and Hills, L.S., 1991, Brute3 and Select: Quickbasic 4 programs for determination of stress tensor configurations and separation of heterogeneous populations of fault-slip data: Computers \& Geosciences, v. 17, p. 23–43.

    Holland, J.H., 1975, Adaptation in Natural and Artificial Systems. Ann Arbor: University of Michigan Press, 183p.
    
    Kozak, A., 2008, Introduction to probability and statistics: applications for forestry and natural sciences: CABI, 2008.
    
    Liesa, C. L., and Lisle, R. J., 2004. Reliability of methods to separate stress tensors from heterogeneous fault-slip data: Journal of Structural Geology, v. 26, p. 559-572. 

    Lisle, R.J., and Srivastava, D.C., 2004, Test of the frictional reactivation theory for faults and validity of fault-slip analysis: Geology, v. 32, p. 569–572.
    
    Lisle, R.J., Orife, T.O., Arlegui, L., Liesa, C., and Srivastava, D.C., 2006, Favoured states of paleostress in the Earth’s crust: evidence from fault-slip data. Journal of Structural Geology v. 28, p. 1051–1066.
    
    Liu, Y., Wu, X., and Shen, Y., 2011, Automatic clustering using genetic algorithms: Applied Mathematics and computation, v. 281, p. 1267-1279.

    Michael, A., 1984, The determination of stress from slip data, faults and folds: Journal of Geophysical Research, v. 89, p. 11517–11526.
    
    Nemcok, M., and Lisle, R. J., 1995, A stress inversion procedure for polyphase fault/slip data sets: Journal of Structural Geology, v. 17, p. 1445-1453. 

    Nemcok, M., Kovac, D., and Lisle, R. J., 1999, A stress inversion procedure for polyphase calcite twin and fault-slip data sets: Journal of Structural Geology, v. 21, p. 597-611.
    
    Pollard, D., D., Saltzer, S., D., and Rubin, A., M., 1993, Stress inversion methods: are they based on faulty assumptions?: Journal of Structural Geology, v. 15, p. 1045–1054.
    
    Ray, A., and Srivastava, D. C., 2008, Non-linear least squares ellipse fitting using the genetic algorithm with applications to strain analysis: Journal of Structural Geology, v. 30, p. 1593–1602.

    Sambridge, M., and Drijkoningen, G., 1992, Genetic algorithms in seismic waveform inversion: Geophysical Journal International, v. 109, p. 323–342.

    Sen, M., Stoffa, P.L., 1995, Global Optimization Methods in Geophysical Inversion: Elsevier Science, B.V, p. 125–157.
    
    Shan, Y., Suen, H., and Lin, G., 2003, Separation of polyphase fault/slip data: an objective-function based on hard division: Journal of Structural Geology, v. 25, p. 829-840.

    Shan, Y., Li, Z., and Lin, G., 2004, A stress inversion procedure for automatic recognition of polyphase fault/slip data sets: Journal of Structural Geology, v. 26, p. 919-925. 

    Shan, Y., and Fry, N., 2005, A hierarchical cluster approach for forward separation of heterogeneous fault/slip data into subsets: Journal of Structural Geology, v. 27, p. 929-936. 

    Shan, Y., and Fry, N., 2006, The moment method used to infer stress from fault/slip data in sigma space: invalidity and modification: Journal of structural geology, v. 28, p. 1208-1213.  

    Shan, Y., Gong, F., Li, Z., and Lin, G., 2007, A grid-search inversion method looking for the best classification of polyhase fault/slip data: Tectonophysics, v. 433, p. 53-64. 

    Spang, J., H., 1972, Numerical method for dynamic analysis of calcite twin lamellae: Geological Society of America Bulletin, v. 83, p. 467–472.

    Wallace, R.E., 1951, Geometry of shearing stress and relation to faulting: Journal of Geology, v. 59, p. 118–130.
    
    Xu, P., 2004, Determination of regional stress tensors from fault–slip data: Geophysical Journal International, v. 157, p. 1316–1330.
    
    Yamaji, A., 2000, The multiple inverse method: a new technique to separate stresses from heterogeneous fault-slip data: Journal of Structural Geology, v. 22, p. 441-452.
    
    Yamaji, A., 2007, An Introduction to Tectonophysics: Theoretical Aspects of Structural Geology. Terrapub, Tokyo, p. 271–282.

    Yin, Z., M., and Ranalli, G., 1995, Estimation of frictional strength of faults from inversion of fault-slip data: a new method: Journal of Structural Geology, v. 17, p. 1327–1335.

    Zajac, B.J., and Stock, J.M., 1997, Using borehole breakouts to constrain the complete stress tensor: Results from the Sijan Deep Drilling Project and offshore Santa Maria Basin, California: Journal of Geophysical Research:, v 102, p. 10083–10100.

    Zang, A., and Stephansson, O., 2010, Stress Field of the Earth’s Crust: Springer, Amsterdam, 319 p.

\end{hangparas}
}