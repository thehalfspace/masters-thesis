\onehalfspacing
The knowledge of stress states in Earth`s crust is a fundamental objective in many tectonic, seismological and engineering geological studies (Engelder, 1993; Lisle et al., 2006; Zang and Stephansson, 2010). Geologists and geophysicists routinely practice methods for determination of the stress tensor from inversion of observations on the stress indicators, such as faults, earthquakes and calcite twin lamellae (Gephart and Forsyth, 1984; Angelier, 1994; Hardebeck and Hauksson, 2001). While the stress inversion is essentially a nonlinear problem, it is commonly solved by linearization methods, such as the direct inversion technique (Angelier, 1979, 1990) and the least squares technique (Michael, 1984). These approaches, however, have several limitations. For example, they oversimplify the problem and become inefficient with increasing nonlinearity due to the dependence of execution on the error surface. Some methods, such as the least squares technique, constrain one of the principal stresses to be vertical (Michael, 1984) and these are best-suited only to Andersonian stress orientation (Anderson, 1951).

An efficient way to solve the inverse problem is by using the nonlinear techniques. Existing approaches for stress inversion are, at best, semi-nonlinear (e. g. Etchecopar et al., 1981; non-linear least squares technique in Angelier et al., 1982; the brute force method in Hardcastle and Hills, 1991). These methods use iterations on an initial model and implement local optimization algorithms that can search the maxima or the minima only within a neighbouring set of possible solutions. By contrast, the global optimization scans complete spectrum of possible solutions for searching the best solution. Xu (2004) elucidates the limitations of the local optimization methods and proposes a hybrid optimization technique that searches the global optima. His method uses a constrained linear inversion and is best suited to relatively simple models. In summary, most existing stress-inversion methods use oversimplified assumptions and/or have a propensity to get entrapped in the first optimum in the neighbourhood of their starting point. Where the first optimum is not the global optimum, these methods may yield incorrect results. In the present study, we develop an elegant method that does not require any linearizing assumption and searches the global best solution efficiently.

We propose a heuristic search technique, the genetic algorithm (GA) (Holland, 1975; Goldberg, 1989; Sen and Stoffa, 1995) that gives the stress tensor by searching the global best solution. The genetic algorithm is superior to the other traditional algorithms in following respects: (i) it does not require any linearizing assumption (e.g., Angelier, 1979; Michael, 1984) (ii) its execution is independent of the error surface, (iii) it is more time-efficient than the unguided searches, such as the exhaustive grid search (e.g., Carey and Brunier, 1974) or the Monte Carlo algorithm (Etchecopar et al., 1981) and (iv) it works well in noisy environments. In the Earth sciences, the application of genetic algorithm has been rather limited, barring a few case studies in seismic waveform inversion (Sambridge and Drijkoningen, 1992), gravimetry (Barker, 1999) and acoustical oceanography (Gerstoft, 1994). In structural geology, the potential of genetic algorithm is yet to be explored, and it has seen its application only in one case of strain analysis (Ray and Srivastava, 2008). We demonstrate a typical application of the GA in stress inversion from fault-slip observations. This approach can be easily extended to inversion from other stress indicators, such as the calcite twin lamellae, earthquake data, and borehole breakouts (Spang, 1972; Gephart and Forsyth, 1984; Zajac and Stock, 1997).